\documentclass[12pt]{article}
\usepackage[utf8]{inputenc}
\usepackage[T1]{fontenc}
\usepackage[russian]{babel}
\usepackage{xcolor}

\begin{document}
\subsection*{Наивный байес и центроидный классификатор}

По определению байесовского классификатора,
необходимо найти $$class(x) =
argmax_y {P(y)\prod_{k=1}^n}P(x^{(k)}|y)$$
Априорная вероятность одинакова для всех классов и равна $P_c$.
Плотность распределения признаков равна
$$P(x^{(k)}|y) = \frac{1}{\sqrt{2 \pi \sigma^2}}
e^{-\frac{x^{(k) - \mu_{yk}}^2}{2\sigma^2}}
\qquad \forall k \in \{ 1, n\}$$
Тогда:
$$ class(x) = argmax_y {P_c
\prod_{k=1}^n}\frac{1}{\sqrt{2 \pi \sigma^2}}
e^{-\frac {({x^{(k)} - \mu_{yk})}^2}{2\sigma^2}} $$
Прологарифмируем рассматриваемое произведение:
$$ L(x, y) = ln(P_c) + ln(\frac{1}{\sqrt{2 \pi \sigma^2}}) n
+ \sum_{k=1}^n {-\frac {({x^{(k)} - \mu_{yk})}^2}{2\sigma^2}} $$
Нахождение $argmax_y(L(x, y))$ эквивалентно нахождению
$argmax_y$ исходной функции.
Рассмотрим только ту часть функции, которая зависит от y
(остальная часть не влияет на $argmax_y(L(x, y))$)
Остается только
$$ \sum_{k=1}^n {-{(x^{(k)} - \mu_{yk}})}^2 = -\rho^2(x, \mu_y) $$
То есть нахождение $argmax_yL(x,y)$ эквивалентно минимизации расстояния от x до $\mu_y$ по всевозможным $y$. Продолжая цепочку эквивалентностей обратно, получаем требуемое утверждение.

\subsection*{ROC-AUC случайных ответов}
Пусть в выборке N элементов, $\alpha$ - доля объектов класса 0.
Соответственно доля объектов класса 1 будет равна $1 - \alpha$.
Покажем, что все зависимости от N, $\alpha$ и $p$, True Positive Rate в среднем будет равен False Positive Rate.
$$ TPR = \frac{TP}{TP + FN} =
\frac{p \alpha N}{p \alpha N + (1 - p) \alpha N} =
\frac{p}{p + 1 - p} = p $$


$$ FPR = \frac{FP}{FP + TN} =
\frac{p (1 - \alpha)N}{p (1 - \alpha)N + (1 - p) (1 - \alpha)N} =
\frac{p}{p + 1 - p} = p $$
Следовательно в среднем ROC кривая будет отрезок (0, 0)-(1, 1).
Поэтому площадь под ней будет равняться в среднем
$ROC AUC = 0.5$.

\end{document}
